\section{Conclusion}
%\vspace{5cm}

We contributed \toolname{}, an interactive system for producing comics about dynamic networks. 
It leverages the form of comics to construct a narrative structure and offers a flexible pen + touch authoring interface for content creation and manipulation. \toolname{} provides automatic transitions and panel recommendations for narrative ideation and accelerated storyboarding. 
% We have initial evidence that people find it easy to learn and use, and are receptive to a tool that enables flexible workflows for data exploration and authoring. 
\rev{We plan to extend our evaluation to study the authoring process in a more longitudinal {\it free-form study}~\cite{ren2018beliv}, focusing on comprehensive evaluation metrics for visualization authoring tools~\cite{amini2018}.}

\section{Acknowledgments}
We would like to thank anonymous reviewers for useful feedback. Nam Wook Kim would like to acknowledge the support from the Kwanjeong Educational Foundation and the Siebel Scholars Foundation.
\newpage
% \toolname{} was designed with four challenges in mind: a need to craft expressive visualization content, dividing this content across panels, producing transitions them, and ultimately conveying a linear narrative. 

% \toolname{} provides various features to manipulate visualization elements such as styling and highlighting, as well as features for laying out panels and transitioning between them. We conducted a study in which graphic designers and data analysts used DataToon to produce comics, and they found it to be both usable and easy to learn. 

% DataToon is available online at [\url{https://datatoon.datacomics.net}].