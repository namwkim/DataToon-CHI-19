\section{Background on Comics (Integrate into Section 2}
\label{sec:background}

In this section, we briefly review existing literature on comics. The goal of this section is to introduce what comics is , what the core components of comics are, and find connections between comics and visualizations for data-driven storytelling, especially to illuminate how we identified our design goals.

\paragraph{Definition} 

McCloud, a comics theorist, defines comics as \textit{Juxtaposed pictorial and other images in deliberate sequence, intended to convey information and/or to produce an aesthetic response in the viewer}~\cite{mccloud1993understanding} or, to put it simply, sequential art ~\cite{mccloud1993understanding,eisner2008comics}. This definition emphasizes the importance of sequences of images (or panels), a narrative \textit{form} specific to comics, and does not dictate any specific type of \textit{content}. It is also what makes comics different from other similar narrative forms like a cartoon or film; i.e., a cartoon is a single image with no sequence while images in a movie are sequential in time not spatially juxtaposed as comics are~\cite{mccloud1993understanding,groensteen2007system}. 


\paragraph{Words and Pictures}

While not emphasized in the definition, another important characteristic of comics is the unique employment of words and pictures. There are many other narrative forms that use combinations of words and pictures. However, the way in which words and pictures interact in comics is distinct from the others~\cite{saraceni2003language}. To this end, McCloud outlines severn different ways to combine words and pictures; e.g., word-specific, picture-specific, and inter-dependent. 

Words in comics are mostly represented as \textit{speech balloons} or \textit{captions}, while often integrated into pictures and treated graphically to convey moods or sounds (e.g, onomatopoeia) ~\cite{eisner2008comics}. On the other hand, pictures can take different levels of representations from realistic to abstract, while words are at the very end of abstraction. The whole map of words and pictures create the vocabulary of comics~\cite{mccloud1993understanding}.

\paragraph{Panel}

A panel is a basic communication unit for comics, serving as an window of information. It encapsulates a portion (e.g., moments or actions) of the whole narrative using combinations of pictures and words therein. Panels are mostly represented as framed rectangles and arranged in an orderly fashion. Each panel represents a delimited space and time in the narrative and controls the duration of attention and affects the pacing of the narration~\cite{caldwell2012comic,duncan2000toward,duncan2015power}. It is only a matter of styles whether panels have different sizes, shapes, border styles (e.g., inset or vignette), or arrangements, but they often can induce different reading experiences. For example, a long stretched panel creates a feeling of a long time span~\cite{eisner2008comics}). 

\paragraph{Gutter and Closure}


Comics are reductive in creation while additive in reading~\cite{duncan2000toward}. Comic artists need to reduce the overall narrative to a set of selected moments and encode the moments into panels. "There is a constant dynamic between what is shown and what could be shown". The panels are spatially juxtaposed to convey both time and space (i.e., temporal mapping), offering a jagged, staccato rhythm of unconnected moments~\cite{mccloud1993understanding}. The white space separating two consecutive panels is called the \textit{gutter} in which readers use their imagination to fill the gap between the isolated moments~\cite{mccloud1993understanding}. This process of reading is called \textit{closure}, the phenomenon of observing the parts but perceiving the whole~\cite{mccloud1993understanding,duncan2015power}. The systematic rules of composing and arranging the panels is what creates the \textit{grammar of comics}~\cite{eisner2008comics}. 

\paragraph{Transitions}

There are a number of types of transitions that can occur between consecutive panels. The most influential taxonomy is McCloud's six types of panel-to-panel transitions, including 1) moment-to-moment: a single action depicted in a series of moments, 2) action-to-action: a single subject (person, object, etc.) portrayed in a series of actions , 3) subject-to-subject: a series of changing subjects in a single scene , 4) aspect-to-aspect: transitions between different aspects of a place, idea or mood, and so forth , 5) scene-to-scene: transitions across significant distances of time and space, and 6) non-sequitur: transitions with no immediate logical connections~\cite{mccloud2011making,mccloud1993understanding}. 

These kinds of transitions can be further categorized based on whether they involve temporal (action-to-action, moment-to-moment), spatial (aspect-to-aspect, scene-to-scene), or spatio-temporal (subject-to-subject, scene-to-scene) shifts between panels~\cite{cohn2003syntatic}. However, some transitions such as scene-to-scene or aspect-to-aspect , often do not clearly show explicit temporal or spatial relations between panels~\cite{cohn2003syntatic}. If panels in transitions are far apart in time or space, it would require more cognitive efforts to generate closure; in such cases, captions or dialogues can be used to aid understanding.


\paragraph{Layouts}

The taxonomy of panel-to-panel transitions does not take into account the global layout of the panels on a page but focuses mainly on the content of adjacent panels~\cite{caldwell2012comic}. The layout is independent of the content of comics, meaning that a sequence of panels can be arranged into numerous layouts with no effect on its meaning~\cite{cohn2014architecture}. However, the perception of the narrative to readers (e.g., pacing, reading order) can be different depending on the physical layout of panels~\cite{cohn2014architecture}. 

A canonical grid is most popularly used~\cite{postema2013narrative,abel2008drawing} while other variations are possible such as staggered, overlapping, and separated panels~\cite{cohn2014architecture}. On the other hand, blockage layouts can confuse readers to deviate from a conventional reading order (left-to-right and down), in which two vertically stacked panels are directly adjacent to a long vertical panel spanning the previous two~\cite{cohn2014architecture}. 

Recent technology advances are beginning to affect how comics are created and presented. For instance, the traditional page layout of panels is often replaced with a long vertical list of panels with scrolling~\cite{goodbrey2013digital}.

Comics as a art form can contain any number of ideas and images~\cite{mccloud1993understanding}. Due to its versatility and familiarity, it has been widely applied in a variety of domains such as product design\footnote{\url{http://www.designcomics.org/}}~\cite{moraveji2007comicboarding}, storyboard~\cite{haesen2010draw}, education~\cite{green2010graphic} and information communication~\cite{tatalovic2009science,caldwell2012information}. 