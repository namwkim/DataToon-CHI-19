\section{Comics for Communicating Data}
Our aim is to leverage the unique form of comics to construct a data story. We briefly examine essential elements in comics to inform the user interface of \toolname{}. 

Comics is a well-established storytelling medium~\cite{mccloud1993understanding,eisner2008comics}, used in many contexts such as storyboarding~\cite{haesen2010draw,moraveji2007comicboarding}, science education~\cite{green2010graphic,tatalovic2009science}, or information communication~\cite{caldwell2012information}. 
McCloud, a comics theorist, describe comics as \textit{juxtaposed pictorial and other images in deliberate sequence, intended to convey information to the viewer}~\cite{mccloud1993understanding} or, to put it simply, \textit{sequential art}. It is different from a movie or animation in which images are sequential in time not spatially juxtaposed. Recently, Bach et al investigated the potential of this genre in communicating data and discussed how data and visualizations are integrated into the comic form to create \textit{data comics}~\cite{bach2017emerging}.


\paragraph{Words and Pictures}


\paragraph{Panel}

\paragraph{Transitions}




\subsection{Comics Essentials}
It's a related work section that deserves attention on  its own. Some contents from the section 4 and some from Nathalie's work with Benjamin.

\subsection{Data Comics}

